\documentclass{article}
\usepackage[utf8]{inputenc}

%%Let's you change margins
\usepackage[left=1in,right=1in,top=1in,bottom=1in]{geometry}

%%Math symbols, proof environments
\usepackage{amsmath,amsthm,amssymb, graphicx}

%%Use this package for matrices
\usepackage{array}

%%Commands for common sets
\newcommand{\R}{\mathbb{R}} %Real numbers
\newcommand{\Z}{\mathbb{Z}} %Integers

\title{Math 61 Worksheet 5 Template} %Remove Template in your title

\author{Sample Student} %Put your name here

\date{\today}

\begin{document}

\maketitle

\subsection*{Problem 1}
Consider ordered strings of 1's and 2's. Let $f_n$ count the number of such strings that sum to $n-1$. For example, $f_3=2$ because there are exactly $2$ ordered strings that sum to $3-1=2$: $11$ and $2$. We also have $f_1=1$ because there is exactly one string that sums to nothing, namely the empty string.
\begin{enumerate}
\item[(a)] Give a combinatorial argument to show that $f_n=f_{n-1}+f_{n-2}$ for $n\geq 2$.
\begin{proof}

\end{proof}
\item[(b)] Give a combinatorial argument to show that $f_{n+2}-1=\sum_{i=1}^{n} f_i$. (Hint: think about strings that sum to $n+2$ and consider the position of the first $2$.)
\begin{proof}

\end{proof}
\end{enumerate}

\subsection*{Problem 2}
The sequence of integers described in Problem 1 is known as the \emph{Fibonacci sequence}. The sequence $f_n$ can be also be defined according to the  recurrence relation $f_n=f_{n-1}+f_{n-2}$,  $n \geq 3$, and the initial conditions $f_1=1$, $f_2=1$.
\begin{enumerate}
    \item[(a)] List the first $10$ Fibonacci numbers.
    \begin{proof}[Answer]
    
    \end{proof}
    \item[(b)] Reprove the identity $f_{n+2}-1=\sum_{i=1}^{n} f_i$ from 1(a) using mathematical induction.
    \begin{proof}
    
    \end{proof}
\end{enumerate}
\subsubsection*{Problem 3}
If $P_n$ denotes the number of permutations of $n$ distinct objects, find a recurrence relation and an initial condition for the sequence $P_1, P_2, P_3, \dots$. 
\begin{proof}[Solution]

\end{proof}
\end{document}
