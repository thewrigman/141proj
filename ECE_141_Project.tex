\documentclass{article}
\usepackage[utf8]{inputenc}

%%Let's you change margins
\usepackage[left=1in,right=1in,top=1in,bottom=1in]{geometry}

%%Math symbols, proof environments
\usepackage{amsmath,amsthm,amssymb, graphicx}

\newcommand{\atan}{\tan^{-1}}

%%Use this package for matrices
\usepackage{array}

%%Commands for common sets
\newcommand{\R}{\mathbb{R}} %Real numbers
\newcommand{\Z}{\mathbb{Z}} %Integers

\title{ECE 141 Project} %Remove Template in your title

\author{Inesh Chakrabarti} %Put your name here

\date{\today}

\begin{document}

\maketitle

\subsection*{Problem 1}
Show that for any desired $\beta$ in $[0, 2\pi)$ there exists a $u$ in $[0, 2\pi)$ so that (5) holds. We can thus treat $\beta$ as the input since for any $\beta$ computed by a controller we can compute the steering angle $u$ via the relation (5) and apply this command to the motor steering for the wheels. This will greatly simplify the equations you have to work with.
\begin{proof}[Solution]
First, let us look at relation (5):
\[\beta = \atan\left(\dfrac{\ell_r}{\ell_r+\ell_f}\tan(u)\right)
\]
Note that we know $\ell_f = 1.1m$ and $\ell_r = 1.7m$, so subsitututing, we are left with 
\[\beta = \atan\left(\dfrac{1.7}{1.7+1.1}\tan(u)\right)= \atan(k\tan(u))
\]
where $k=\dfrac{1.7}{2.8} \approx 0.607142857$ \newline
Note that the range of $tan(u)$ for $u \in [0, 2\pi)$ is all real numbers, so the range of $ktan(u)$ must also be all real numbers. Therefore, we can deal with $\atan$ function's regular range, as we have included all numbers in its domain. Assuming that the given $\atan$ refers to a four quadrant tangent inverse function, we thus arrive at the conclusion that for any desired $\beta \in [0, 2\pi)$ there exists some $u \in [0, 2\pi)$
\end{proof}

\subsection*{Problem 2}
We first consider the lane keeping problem, i.e., the design of a controller that keeps the car in the center of its lane. For this purpose we assume the car’s velocity to be constant at 35 mph, that the lane center corresponds to $y = 0$, and that we have a sensor measuring $y$ (in reality, the position of the car on the lane would be computed by using vision to detect the location of the lane markers). Linearize the equations of motion and design a controller that stabilizes the car at $y = 0$ using the linearized model (use the transfer function from $\beta$ to $y$). You don’t need to work with equation (1) since $x$ will not be at equilibrium. Provide some plots showing the controller works as intended.
\begin{proof}[Solution]

First we note that the nonlinear differential equation we have is
\end{proof}
\end{document}
